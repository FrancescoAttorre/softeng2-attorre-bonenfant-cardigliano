\renewcommand{\thesubsection}{\Alph{subsection}}
In the developed software all the main features and functions of CLup have been implemented, leaving aside additional features that could be easily added at a later time, on the base structure already existent.
We present below a list of the implemented features:
\begin{itemize}
	\item Registration of an Activity to CLup, after providing username, password and P.IVA, and log in of such Activity in order to insert its buildings.
	\item Buildings insertion, providing name, address, a capacity, opening times and optionally departments with surplus of capacity.
	\item Generation of a building access code for each building, to allow store managers to sign up for a building.
	\item Registration of customers after providing username and password, in order to make a booking.
	\item Login of customers after providing correct credentials.
	\item Providing clients a list of buildings among which to choose the one to queue or book for, avoiding buildings too far from the customer.
	\item Provide an estimated travel time depending on the means of transport chosen by the customer, using an external maps service.
	\item Computation of estimated waiting times in queue for each line up ticket, basing on building's statistics.
	\item Discovery requests to keep the estimated waiting time as updated as possible.
	\item Giving physical customers the possibility to acquire a line up ticket having the store manager as an intermediary.
	\item Queue management.
	\item Use of departments for each building with additional capacity in order to increase the number of people allowed if directed only to those departments.
	\item Acquisition of a booking digital ticket, after providing a date, a slot of time, a permanence time and optionally departments.
	\item Provide a list of available time slots, multiple of 15 minutes, to a customer who wish to book for a building.
	\item Providing a positive/negative result after a check by the store manager to decide whether a customer can enter or not.
	\item Making tickets become expired 10 minutes after their validation.
	\item Allowing store managers to register customer's exits.
\end{itemize}
Other features have been discarded since they have been considered of minor importance. They are for instance:
\begin{itemize}
	\item Notification to leave for a building when TravelTime <= WaitingTime, this feature can be developed by the client, once having the information provided by the server and by the external maps service.
	\item Alternatives of time slots or buildings, which is an additional feature, not one of the main ones necessary for the application to work properly. 
	\item Generation of QR codes with ticket's informations.
	\item Computation of the estimated permanence time for registered customers, basing on their previous visits to the building, additional feature.
\end{itemize}
All these features are suggested for a better effectiveness of the application, except the generation of QR code which is particularly critical for the practical use.