\renewcommand{\thesubsection}{\Alph{subsection}}
\subsection{Introduction}
The goal of this document (I\&T: Implementation and Testing) is to describe the structure of the implementation of CLup, together with the main frameworks adopted for its development. It also includes informations about how the testing plan. This document is preceded by the RASD (Requirements Analysis and Specification Document) and DD (Design Document) documents. In particular, the implementation has been driven by the decisions made in the Design Document, about the main components, interfaces and architectural patterns to exploit, and also implementation, integration and testing plans.
\subsection{Scope}
This document illustrates how the implementation of CLup system has been carried on. CLup is an application which aims to avoid the occurrence of crowds in a period of global pandemic. To achieve this goal, it allows customers to queue before entering a building without creating long physical queues in front of it, or to book a visit to access a building, through an application. CLup will allow people to save time and be safer, managing both the accesses of people who use the application directly, and of people who go physically to the store, using store managers as intermediaries. CLup can be accessed by activities, which will register their buildings giving all the necessary information, by store managers who manage the entrances to the building they're related to, and by customers. Booking a visit, customers can also decide how much time they want spend in the building and, optionally, which specific departments they want to visit, in order to allow a greater number of people to enter when possible.
\newpage