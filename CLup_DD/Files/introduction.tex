\renewcommand{\thesubsection}{\Alph{subsection}}
     	\subsection{Purpose}
	The goal of this Design Document is to provide a general view of the architecture of the software, adding more technical details to the information provided in the RASD document. The architecture will be explained in terms of hardware and software components, the interfaces they provide and the interactions between them. The document is going to face also the integration and testing plans, and the main design patterns to be exploited. Information related too much on the implementation will not be dealt in detail since this document aims to the description of an high level architecture of the system. 
         \subsection{Scope}
         CLup is an application which aims to manage crowds in a period of global pandemic. It allows store managers to monitor the entrances in a building and to avoid the the formation of queues in front of stores. CLup will allow people to save time and be safer, managing both the accesses of people who use the application directly, and of people who go physically to the store, using store managers as intermediaries. CLup can be accessed by customers, store managers and activities which will register stores available to be visited. Customers using the application, moreover, will be notified when they're supposed to leave for the store, and, when registered, they will have the possibility to book a visit to a store, being informed on alternative time slots or similar stores when nothing is available for their choices.  Booking a visit, customers can decide how much time they want spend in the building and, optionally, which specific departments they want to visit, in order to allow a greater number of people to enter when possible.
         
         \subsection{Definitions, Acronyms, Abbreviations}
         	\subsubsection{Definitions}
		\begin{itemize}
			\item \textcolor{BrickRed}{User}: with this term we refer to StoreManagers, AppCustomer, Activities which are the ones who use the mobile or web application  and its services. They will be associated to an ID in order to be recognized by the system.
		\end{itemize}
		
		\subsubsection{Acronyms}
		\begin{itemize}
			\item \textcolor{BrickRed}{RASD}: Requirements Analysis and Specification Document 
			\item \textcolor{BrickRed}{GPS}: Global Positioning System
			\item \textcolor{BrickRed}{CLup}: Customers Line-up
			\item \textcolor{BrickRed}{DD}: Design Document 
			\item \textcolor{BrickRed}{HTTP}: HyperText Transfer Protocol
			\item \textcolor{BrickRed}{TLS}: Transport Layer Security
			\item \textcolor{BrickRed}{JSON}: JavaScript Object Notation
			\item \textcolor{BrickRed}{DBMS}: Database Management System
			\item \textcolor{BrickRed}{JDBC}: Java DataBase Connectivity
			\item \textcolor{BrickRed}{API}: Application Programming Interface
			\item \textcolor{BrickRed}{REST}: REpresantional State Transfer
		\end{itemize}
		
		\subsubsection{Abbreviations}
		\begin{itemize}
			\item \textcolor{BrickRed}{Rn}: requirement number n
			\item \textcolor{BrickRed}{Cn}: component number n
		\end{itemize}
		
	\subsection{Revision History}
	In a first version of the document, a major number of components were created, but then has been considered more appropriate to group them into a few main macro-components with different functions within them.
	
	\subsection{Reference Documents}
	\begin{itemize}
			\item \textcolor{BrickRed}{Specification Document}: “R\&DD Assignment AY 2020-2021.pdf”
			\item \textcolor{BrickRed}{Slides of the lectures}
			\item \textcolor{BrickRed}{UML diagrams}: https://www.uml-diagrams.org/
			\item \textcolor{BrickRed}{Database Administration}: https://galeracluster.com/library/documentation/deployment-variants.html
	\end{itemize}
		
	\subsection{Document Structure}
	This DD in composed by 7 main sections:
	\begin {itemize}
	 	\item SECTION 1 is the introduction, containing the scope and the purpose of the system, together with Definitions, Acronyms, Abbreviations, the revision history of the document deployment, the reference documents and the document structure.
		\item SECTION 2 contains the architectural design of the system, described in terms of components, runtime view with sequence diagrams showing the way the various components interact with each other, the deployment view, component interfaces and the related class diagram and finally an overview of patterns and architectural styles used, plus other design decisions.
		\item SECTION 3 contains the user interface design, in particular some mockups show the main mobile and web application interfaces.
		\item SECTION 4 contains requirements traceability showing how the requirements described in the RASD map the design components identified in this document. This table clearly shows if all the components cover at least one requirement and if each requirement is met by at least one component.
		\item SECTION 5 concerns the implementation, integration and testing. Here it is defined how the subcomponents should be implemented and integrated and which kinds of tests should be carried out on them.
		\item SECTION 6 contains a table with the effort spent by each member of the group.
		\item SECTION 7 tools used.
	\end{itemize}
	
