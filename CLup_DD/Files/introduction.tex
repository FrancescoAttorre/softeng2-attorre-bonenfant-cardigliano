\renewcommand{\thesubsection}{\Alph{subsection}}
     	\subsection{Purpose}
	The goal of this document (DD: Design Document) is to

         \subsection{Scope}
         
         \subsection{Definitions, Acronyms, Abbreviations}
         	\subsubsection{Definitions}
		%\begin{itemize}
		%\end{itemize}
		
		\subsubsection{Acronyms}
		\begin{itemize}
			\item \textcolor{BrickRed}{RASD}: Requirements Analysis and Specification Document 
			\item \textcolor{BrickRed}{GPS}: Global Positioning System
			\item \textcolor{BrickRed}{CLup}: Customers Line-up
			\item \textcolor{BrickRed}{DD}: Design Document 
		\end{itemize}
		
		\subsubsection{Abbreviations}
		\begin{itemize}
			\item \textcolor{BrickRed}{Rn}: requirement number n
		\end{itemize}
		
	\subsection{Revision History}
	
	\subsection{Reference Documents}
	\begin{itemize}
			\item \textcolor{BrickRed}{Specification Document}: “R\&DD Assignment AY 2020-2021.pdf”
			\item \textcolor{BrickRed}{Slides of the lectures}
	\end{itemize}
		
	\subsection{Document Structure}
	This DD in composed by 7 main sections:
	\begin {itemize}
	 	\item SECTION 1 is the introduction ...
		\item SECTION 2 contains the architectural design
		\item SECTION 3 with the user interface design
		\item SECTION 4 contains requirements traceability showing how the requirements described in the RASD map the design components identified in this document.
		\item SECTION 5 concerns the implementation, integration and testing. Here it is defined how the subcomponents should be implemented and integrated and which kinds of tests should be carried out on them.
		\item SECTION 6 contains a table with the effort spent by each member of the group.
		\item SECTION 7 references/tools.
	\end{itemize}
	
